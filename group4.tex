\documentclass[a4paper,10pt]{article}
\usepackage{url}
\usepackage[utf8]{inputenc}
\usepackage[backend=biber,autocite=footnote,notetype=foot+end,style=authortitle-ibid]{biblatex}
\bibliography{group4.bib}

\begin{document}

\title{Sentiment Analysis of Twitter Data based on Implicit Categories in Textual Weather Data}
\author{
    Johnston, Cian
    \and
    Ravindran, Aishwarya
    \and
    Chavady, George
    \and
    Karode, Sameer
    \and
    Deepak, Shravani Kulkarni
}

\maketitle
\section{Introduction}

Sentimental mining is a critical and essential area because sentiment fundamentally relates to a person’s emotions, impression and attitude. Analysing every individual’s sentiments from a text analytics point of view accurately is challenging. We are basically trying to comprehend the thoughts and assumptions of an individual regarding a concept or topic with some context be it positive, negative or neutral \footfullcite{aylien_sentiment}. It is estimated that nearly 2.5 quintillion bytes of data is generated each day \footfullcite{blazon_howmuchdata} and there is a plethora of information relating to a person’s messages, tweets, documents, emails, chats, conversations and comments available. Sentiment analysis aids us in analysing this huge amount of data efficiently.

In this paper, we focus on performing sentiment analysis on the weather based on data retrieved from Twitter \footfullcite{mining_twitter_data}. Twitter, being an acclaimed stage for social networking possess and people to express their interests and thoughts has plenty of information from posts known as “tweets”. With over 500 million tweets per day pertaining to almost anything, it is an excellent platform due to its accessibility and real-time analysis of the data which is crucial for sentiment analysis. It is not uncommon that weather plays an important role in an individual’s mood and its consequences because of the decisions made. [I need some help in writing what exactly we plan to do with the weather data… for instance determine if the person is happy or not by performing sentiment analysis on tweets regarding weather? Basically, we need to write something that sets us apart from all the previous work done using these 3 even if its simple (sentiment analysis, twitter data and weather)] 

In its preliminary stage, we [need to write what we have done so far].

\section{Related Work}

As the accessibility to real-time information or otherwise is increasing, performing analysis on this data is also becoming viable. There has been a lot of notable research done on sentiment analysis of twitter data in general and studies emphasizing on correlating weather with people’s mood and sentiments from Twitter data by measuring temperature, humidity and atmospheric pressure \footfullcite{park2013mood}. These weather variables seem to have a satisfactory impact on one’s mood. Another paper \footfullcite{bagheri2017sentiment} focuses on implementing sentiment analysis on Twitter data using the Twitter API’s and a wealth of available libraries. Twitter is known for having small texts and abbreviations used by people in their posts or tweets making it challenging to extract polarity of the texts and hence researchers resort to utilizing deep learning and machine learning techniques.
Researchers have performed sentiment analysis for various reasons and different areas like elections, politics, movie ratings and fashion to name a few. In 2015, there was a noteworthy research done in the vision of predicting future crime on each area of a major city, Chicago, Illinois of the United States using GPS tagged twitter data \footfullcite{chen2015crime}. They aimed to predict the time and location during which a specific type of crime is expected to occur by applying lexicon-based sentiment analysis on categorized weather data combined with kernel density estimation of historical crime incidents and were successful. Hannak et al. \footfullcite{hannak2012tweetin} concentrate on using a Twitter specific sentiment extraction methodology and explore a corpus of over 1.5 billion tweets. With the help of machine learning techniques on Twitter corpus correlated with the weather at a particular time and location of the tweets, it was concluded that aggregate sentiment follows different climate and seasonal patterns.

\section{Methodology}

\subsection{Twitter Data}

Twitter provides an API for developers to both read data and interact with users. This has a number of limitations, including rate limits \footfullcite{twitter_api_docs_rate_limit}, and a requirement to request an API key. In order to access a useful volume of data, this would require a large number of HTTP requests to Twitter. Thankfully, the Internet Archive provides a large dataset of posts on Twitter for the year of 2018 in TAR format \footfullcite{archiveorg_twitter}. While these datasets are essentially a subset of the \textit{global} content of Twitter, and technically much larger than required, they are hosted using BitTorrent and are thus much more straightforward to download. For the purposes of this paper, we elected to use a subset of the Twitter archive data from the year 2018.

In any case, these large archives need to be preprocessed before any meaningful analysis can be performed. To this end, a small program was written to consume the entire TAR archive and filter out Tweets matching certain criteria, serializing a subset of the data to CSV format. This was done to avoid extracting the entire archive to disk, which is a time-consuming operation. The following criteria were used for extracting Tweets of interest:
\begin{itemize}
    \item{
        Having geo-location information within Ireland, or
    }
    \item{
        Posted by a user with user-specified location containing the string \textit{Ireland}.
    }
\end{itemize}

Note that not all Tweets posted by users in Ireland will necessarily have geo-location information attached, and not all Tweets posted by users claiming to be located within Ireland are necessarily so.

\subsection{Weather Data}

Some historical weather data was collected from MET \'{E}ireann, the Irish National Meteorological Service. \footfullcite{metie}. Unfortunately, historical textual weather forecast data is not available from MET \'{E}ireann; to work around this, we used the Internet Achive's Wayback Machine \footfullcite{wayback_machine} to access previously published versions of the MET \'{E}ireann homepage which contains daily textual forecast data.

To access previously versions of the website more conveniently, the tool \texttt{wayback-machine-scraper} was utilised \footfullcite{github_sangaline_wayback-machine-scraper}. This is a command-line utility that interfaces with \textit{Wayback Machine} and allows a user to download a number of snapshots of a website for a specified date range. For the purposes of this paper, we fetched all the saved snapshots of the MET \'{E}ireann homepage for the year of 2018. Note that this is a sparse dataset, and snapshots of this data is not available for every day.

As an alternative source of textual forecast data, we turned to a more unothodox source -- Boards.ie is a discussion board with a wide range of fora which, naturally, includes the topic of weather. One particular thread of interest on this sub-forum has almost daily forecasts provided by an amateur meterologist with the moniker 'M.T. Cranium' \footfullcite{boards_ie_mt_cranium_forecasts}. The relevant print versions of the thread spanning the year of 2018 were saved, and the relevant daily forecasts were extracted using a Python script. 


\section{Results}

Results go here.

\section{Conclusion}

Conclusion goes here.

\end{document}
